% PhD thesis template
% © Jan-Philipp Kappmeier, Apache License 2.0
%%%%%%%%%%%%%%%%%%%%%%%%%%%%%%%%%%%%%%%%%%%%%%%%%%%%%%%%%%%%%%%%%%%%%%%%%%%%%
% General text definitions.
% This contains commands that form text.
%
% Use to define typographic abbreviations.
\usepackage{xspace}
% Use to do some fancy string manipulation
\usepackage{xstring}

\usepackage{xargs}

%%%%%%%%%%%%%%%%%%%%%%%%%%%%%%%%%%%%%%%%%%%%%%%%%%%%%%%%%%%%%%%%%%%%%%%%%%%%%
% Abbreviations.
%
% Some commands that are common abbreviations, similar to the \iif command.
% Some commands that are often used in scientific/mathematical writing
% are absolutely *never* to be used in written documents! (They are ok on
% the white/black board or in lecture scripts.)
%
% Some other abbreviations are allowed to be used in formal texts. The
% following definitions take care for correct spacing. If the abbreviations
% makes sense to be used at the start of a sentence, they are available in
% capital and minor initial letters.
\newcommand{\ie}{i.\nolinebreak[4]\hspace{0.125em}\nolinebreak[4]e.,\@\xspace}
\newcommand{\Ie}{I.\nolinebreak[4]\hspace{0.125em}\nolinebreak[4]e.,\@\xspace}
\newcommand{\eg}{e.\nolinebreak[4]\hspace{0.125em}\nolinebreak[4]g.,\@\xspace}
% The following abbreviations should *never* be used in written documents!
% \newcommand{\wlogk}{...}
% \newcommand{\Wlog}{...}
% \newcommand{\wrt}{...}


\newcommand{\problemidx}[1]{\index{#1@\textsc{#1}}}


%%%%%%%%%%%%%%%%%%%%%%%%%%%%%%%%%%%%%%%%%%%%%%%%%%%%%%%%%%%%%%%%%%%%%%%%%%%%%
% Mathematical problems.
\newcommand{\problemDef}[4]{%
  \StrGobbleLeft{#1}{1}[\lastPart]%
  \StrLeft{#1}{1}[\firstLetter]%
  \begin{center}%
    \setlength{\mathindent}{0cm}%
    \framebox{%
      \parbox{0.8\textwidth}{%
      \begin{tabular}{p{0.105\textwidth}p{0.65\textwidth}@{}}%
        {\sffamily Problem:} & {\ \hspace{-0.3333em}\sffamily\color{\mainColor}#1\problemidx{#1 Problem}}\\[2mm]%
        \emph{Instance}: & #2\tabularnewline[2mm]%
        \emph{Task}:     & #3%
      \end{tabular}%
      \ifstrempty{#4}%
        {}%
        {%
          \setlength{\mathindent}{\standardIntend}\\[2mm]%
          #4%
        }%
      }%
    }%
  \end{center}%
}%


%%%%%%%%%%%%%%%%%%%%%%%%%%%%%%%%%%%%%%%%%%%%%%%%%%%%%%%%%%%%%%%%%%%%%%%%%%%%%
% Index definitions.
% The \idx command writes a word and at the same time adds it to the index 
\newcommand{\idx}[1]{\index{#1}#1}
% The \eidx writes a word and at the same time uses a different word for the
% index. Use if a word is used in a different grammatical form than used in
% the index.
\newcommand{\eidx}[2]{\index{#2}#1}
% Highlights a word using \emph and places it in the index
\newcommand{\emphidx}[1]{\index{#1}\emph{#1}}
% Highlights a word and uses a different word for the index. See \eids
\newcommand{\empheidx}[2]{\index{#2}\emph{#1}}
% Add own index versions here, for definitions, problems, different styles
