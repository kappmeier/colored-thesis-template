% PhD thesis template
% © Jan-Philipp Kappmeier, Apache License 2.0
%%%%%%%%%%%%%%%%%%%%%%%%%%%%%%%%%%%%%%%%%%%%%%%%%%%%%%%%%%%%%%%%%%%%%%%%%%%%%%%%
% XeLaTeX allows to use all types of Open Type fonts. In the following a font
% setup using publicly available fonts from Adobe is defined, together with a
% comercially available math font fitting well to the text fonts.
% Uncomment to enable the fonts.

\usepackage[log-declarations=false]{xparse}

% Loading fontspec and math fonts allowing for Unicode. Has to happen in this order.
\usepackage{amsmath}
\usepackage[quiet]{fontspec}
\usepackage{unicode-math}

% Define Adobe fonts and fitting math font Typoma Minion Math.
\setmainfont[Ligatures=TeX]{Minion Pro} % available with an acrobat reader installation
\setsansfont[Ligatures=TeX]{Myriad Pro} % available with an acrobat reader installation
\setmathfont{Minion Math} % available on typoma.com

% The math fonts does not include (Bold) Calligraphic styles, as backup we use XITS package.
% https://fontlibrary.org/de/font/xits-math
\setmathfont[range={\mathcal},StylisticSet=1]{XITS Math}
\setmathfont[range={\mathbfcal},StylisticSet=2]{XITS Math}

% Additional font families
\newfontfamily\semiboldsans[Ligatures=TeX]{Myriad Pro Semibold}
\newfontfamily\boldsans[Ligatures=TeX]{Myriad Pro-Bold}
\newfontfamily\semiboldserif[Ligatures=TeX]{Minion Pro Semibold}

%%%%%%%%%%%%%%%%%%%%%%%%%%%%%%%%%%%%%%%%%%%%%%%%%%%%%%%%%%%%%%%%%%%%%%%%%%%%%%%%
% Non existing small caps.
%
% Some fonts do not have small caps (either in general or fore some font shapes.
% Mostly, italic small caps are not available. However, the Adobe Minion Pro and
% Myriad Pro fonts to not contain small caps in general.
% The following commands redefine the font shapes to be substituted by other
% shapes when used.
% This means, they are simply not small caps, but the standard font. If small
% caps are really needed, there are two options:
%  1. Use a font that is supporting it
%  2. Use faux small caps: https://tex.stackexchange.com/a/225078/47921
%     (Fake small caps with XeTeX/fontspec?)
%
% If the commands fail, see the next section.
%
% For classic only
%\DeclareFontShape{TU}{lmr}{m}{scit} { <-> ssub * lmr/m/sc }{}
%
% For Adobe fonts only
% MyriadPro does not have small caps, neither normal nor italic. A pity.
\DeclareFontShape{TU}{MyriadPro(0)}{m}{scit} { <-> ssub * MyriadPro(0)/m/sc }{}
\DeclareFontShape{TU}{MyriadPro(0)}{m}{sc} { <-> ssub * MyriadPro(0)/m/n }{}
\DeclareFontShape{TU}{MyriadProSemibold(0)}{m}{sc} { <-> ssub * MyriadProSemibold(0)/m/n }{}

%%%%%%%%%%%%%%%%%%%%%%%%%%%%%%%%%%%%%%%%%%%%%%%%%%%%%%%%%%%%%%%%%%%%%%%%%%%%%%%%
% Errors with the redeclaration of fonts may occur, e.g. when the font encoding
% is wrongly specified. The following command definition \showfont prints
% information about the given font, which should be used in the new font
% declaration as parameters.
%
% The source is  https://tex.stackexchange.com/a/56097/47921
% (Use of \scshape or \textsc with fontspec/xelatex)
%
% Usage examples:
%  {\scshape \showfont Aa}
%  {\fontspec[Letters=SmallCaps]{Myriad Pro Semibold} \showfont Aa}
%  {\semiboldsans\textsc \showfont Aa}
%  {\bfseries\scshape \showfont Aa}
%
\makeatletter
\newcommand{\showfont}{Encoding: \f@encoding{},
  Family: \f@family{},
  Series: \f@series{},
  Shape: \f@shape{},
  Size: \f@size{}
}
\makeatother
