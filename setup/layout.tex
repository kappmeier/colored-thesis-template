% PhD thesis template
% © Jan-Philipp Kappmeier, Apache License 2.0
%%%%%%%%%%%%%%%%%%%%%%%%%%%%%%%%%%%%%%%%%%%%%%%%%%%%%%%%%%%%%%%%%%%%%%%%%%%%%
% The main colors of the thesis.
% If the thesis should not be completely colored (as this is expensive in
% printing), set the mainColor to black, but leave first figureColor with the
% original value if figures should still contain colors.
%
% The main color for headlines, the tile, paragraphs and so on.
\newcommand{\mainColor}{MidnightBlue}
%
% The mainly used color for figures. Is \mainColor, if that is not black.
\newcommand{\figureColor}{\mainColor}
%
% Different name in the ABCD-scheme for the primary color.
\newcommand{\figureColorA}{\figureColor}
%
% Secondary image color.
\newcommand{\figureColorB}{YellowGreen}
%
% The ternary image color.
\newcommand{\figureColorC}{RedOrange}
%
% The quaternary image color.
\newcommand{\figureColorD}{Fuchsia}
%
% A last pre-defined color, gray which works well with all color schemes.
\newcommand{\flatColor}{gray}


%%%%%%%%%%%%%%%%%%%%%%%%%%%%%%%%%%%%%%%%%%%%%%%%%%%%%%%%%%%%%%%%%%%%%%%%%%%%%%%%
% Include the hyperref package. This allows the pdf file to have clickable
% links. The links may go to external or to internal targets. Internal links are
% used for references to figures or algoriths, to the bibliography, or others.
% External links are URLs.
% The links are black (as default font color), with the exception that linkcolor
% which links to internal anchors (Figures, Chapters, etc) have the flat color.
\usepackage
[
  unicode,
  pdfencoding=auto,
  breaklinks=true,
  hyperfigures,
  hidelinks,
  colorlinks=true,
  linkcolor=\flatColor,
  urlcolor=black,
  citecolor=black,
  hyperindex,
  bookmarksnumbered,
  plainpages=false,
  pdftitle={\thesistitle},
  pdfauthor={\thesisauthor},
  pdfsubject={\thesissubject},
  pdfcreator={xelatex}, 
  pdfkeywords={\thesiskeywords}
]{hyperref}

%%%%%%%%%%%%%%%%%%%%%%%%%%%%%%%%%%%%%%%%%%%%%%%%%%%%%%%%%%%%%%%%%%%%%%%%%%%%%%%%
% Paragraph formatting.
% This style sets a paragraph to be semi bold in the main color. It does not
% look to heavy. Also, paragraph headlines are not on their own line but end
% with a . (dot) and after a shor space the paragraph text begins. Each
% paragraph will have a small skip space above. It is intended that paragraphs
% are used extensively.
\makeatletter
\renewcommand\paragraph{\@startsection{paragraph}{4}{0mm}%
{-\baselineskip} % beforeskip
{-\baselineskip} % afterskip
{\color{\mainColor}\semiboldsans}
}%
\makeatother 
\let\originalparagraph\paragraph
\renewcommand{\paragraph}[2][.]{\originalparagraph{#2#1}}
\let\originalsubparagraph\subparagraph
\renewcommand{\subparagraph}[2][.]{\originalsubparagraph{#2#1}}

%%%%%%%%%%%%%%%%%%%%%%%%%%%%%%%%%%%%%%%%%%%%%%%%%%%%%%%%%%%%%%%%%%%%%%%%%%%%%%%%
% Publication remark.
% In thesises it required that content, that is already published elsewere is
% disclosed. Publication remark adds a paragraph with the text
% 'Publication Remark:', which in contrast to the default paragraph does not
% end with a dot.
% It is supposed to be used in the abstract at the beginning of each chapter,
% as defined by the command \chapterabstract (see below)
\newcommand{\publicationremark}{\originalparagraph{Publication Remark:}}

\setkomafont{sectioning}{\sffamily\color{\mainColor}}
\addtokomafont{subsubsection}{\semiboldsans}
\setkomafont{subparagraph}{\normalfont\sffamily\itshape}

\setlength{\mathindent}{1.5cm}
\newlength{\standardIntend}
\setlength{\standardIntend}{\mathindent}
\setlength{\mathindent}{\standardIntend}

\renewcommand{\headfont}{\color{\mainColor}\footnotesize\sffamily} % font for the header
\renewcommand{\pnumfont}{\color{\mainColor}\footnotesize\sffamily} % font for the pagenumbers


% Dictum text before the main part starts.
\makeatletter
\renewcommand{\dictum}[2][]{\par%
  \begingroup
    \raggeddictum\parbox{\dictumwidth}{%
      {\@dictumfont{\raggeddictumtext #2\strut\par}%
        \def\@tempa{#1}\ifx\@tempa\@empty\else%
          {\raggeddictumtext\vskip-1ex\par}%
          \raggeddictumauthor\@dictumauthorfont\dictumauthorformat{#1}%
          \strut\par%
        \fi%
      }%
    }\par%
  \endgroup
}
\makeatother

\renewcommand*{\dictumwidth}{.6667\textwidth}
\setkomafont{dictumtext}{\normalfont\normalcolor\small}
\renewcommand*{\dictumauthorformat}[1]{\vspace{5mm}-- #1}


% Short abstracts at the beginning of chapters.
\newcommand\abstractname{Abstract} %%% here
\makeatletter
\newenvironment{chapterabstract}{%
\small
\begin{addmargin*}{2cm}%
}
{
\end{addmargin*}%
\chapterheadendvskip
}
\makeatother
