% PhD thesis template
% © Jan-Philipp Kappmeier, Apache License 2.0
%%%%%%%%%%%%%%%%%%%%%%%%%%%%%%%%%%%%%%%%%%%%%%%%%%%%%%%%%%%%%%%%%%%%%%%%%%%%%
% Basic setup
% line spacing
\usepackage[dvipsnames]{xcolor}

% have klickable urls in pdf
\usepackage{url}

% include other pdf files. either for front and rear cover, but also for images
\usepackage{pdfpages}

% Do calculations
%\usepackage{calc}

%%%%%%%%%%%%%%%%%%%%%%%%%%%%%%%%%%%%%%%%%%%%%%%%%%%%%%%%%%%%%%%%%%%%%%%%%%%%%%
% Index
\usepackage{makeidx}
\renewcommand{\indexname}{Subject Index}



%%%%%%%%%%%%%%%%%%%%%%%%%%%%%%%%%%%%%%%%%%%%%%%%%%%%%%%%%%%%%%%%%%%%%%%%%%%%%%
% Set up bibliography. This section sets up both, the technical properties,
% i.e. the packages to load and the files to read the references from, and
% also the style of the bibliography.
%
% Use biber for setting up the bibliography instead of bibtex. Software like
% kile will detect automatically whether it should run biber or bibtex.
% Otherwise you have to use Biber instead of BibTex manually in your build
% process.
% The template can be used for bibtex and biber.
% If missing either biber or biblatex.sty, use the following command.
% apt-get install texlive-bibtex-extra biber
\usepackage[
  backend=biber, % Use biber
  style=trad-alpha, % Citation like Abc88, the style in CS or Math.
  url=false, % Do not show urls for non-web references, books, paper, etc.
  doi=false, % Do not show DOI
  isbn=false, % Do not show ISBN numbers
]{biblatex}
% Disable also language field
\AtEveryBibitem{\clearlist{language}} % clears language
%
% add a colon after "in" for incollections, inbook, ...
\renewbibmacro{in:}{\printtext{In:\intitlepunct}} 
% use and before the final name of a list
\renewcommand*{\finalnamedelim}{\addspace \bibstring {and}\space}
% Reformat the volume number for collections
\DeclareFieldFormat[incollection]{volume}{Volume #1 of}
% Reformat the series in an incollection bibliography entry
\DeclareFieldFormat[incollection]{series}{\mkbibemph{#1}}
% Format the edition field to use typographic numbers
\usepackage[super]{nth}
\DeclareFieldFormat{edition}{%
  \ifinteger{#1}
    {\printtext{\nth{#1} edition}}
    {\MakeLowercase{#1}~\bibstring{}}
}
% General style for a bibliography entry for the
% volume, edition, number and pages section.
\renewbibmacro*{volume+number+pages+eid}{%
  \printfield{volume}%
  \setunit*{\volumenumberdelim}%
  \printfield{number}%
  \setunit{\addcolon\addspace}
  \printfield{pages}%
  \newcommaunit%
  \printfield{eid}%
}%
% Declare the style for online references. This is mostly a copy from
% the bibtex source code with some modification. You probably can remove it
\DeclareBibliographyDriver{online}{%
  \usebibmacro{bibindex}%
  \usebibmacro{begentry}%
  \usebibmacro{author/editor+others/translator+others}%
  \setunit{\labelnamepunct}\newblock
  \usebibmacro{title}%
  \newunit
  \printlist{language}%
  \newunit\newblock
  \usebibmacro{byauthor}%
  \newunit\newblock
  \usebibmacro{byeditor+others}%
  \newunit\newblock
  \printfield{version}%
  \newunit
  \printfield{note}%
  \newunit\newblock
  \printlist{organization}%
  \newunit\newblock
  \iftoggle{bbx:eprint}
    {\usebibmacro{eprint}}
    {}%
  \newunit\newblock
  \usebibmacro{date}%
  \newunit\newblock
  \usebibmacro{url+urldate}%
  \newunit\newblock
  \usebibmacro{addendum+pubstate}%
  \setunit{\bibpagerefpunct}\newblock
  \usebibmacro{pageref}%
  \newunit\newblock
  \iftoggle{bbx:related}
    {\usebibmacro{related:init}%
     \usebibmacro{related}}
    {}%
  \usebibmacro{finentry}}

\DeclareFieldFormat{titlecase}{#1}

% Use the default color to format certain parts of a bibliography entry.
\newcommand{\bibtitlehighlight}[1]{\color{\mainColor}\semiboldserif{#1}}

% We have disabled printing of url, isbn, and doi in the format specification above.
% However, when the bibliography entry in the references file contains such information,
% make the entry clickable by linking to the url.
% The following does the following:
% When a DOI is given, this is used as it is the proper method to link to scientific documents
% When no DOI is present, but an URL, the URL is used.
% When only ISBN or ISSN is present, a link to Google book search is generated and used
\newbibmacro{string+doiurlisbn}[1]{%
  \iffieldundef{doi}{%
    \iffieldundef{url}{%
      \iffieldundef{isbn}{%
        \iffieldundef{issn}{%
          {\bibtitlehighlight{#1}}
        }{%https://www.google.com/search?tbo=p&tbm=bks&q=issn:1052-6234
          \href{https://www.google.com/search?tbo=p&tbm=bks&q=issn:\thefield{issn}}{\bibtitlehighlight{#1}}%
        }%
      }{%https://www.google.com/search?tbo=p&tbm=bks&q=isbn:9783642059148
        \href{https://www.google.com/search?tbo=p&tbm=bks&q=isbn:\thefield{isbn}}{\bibtitlehighlight{#1}}%
      }%
    }{%
      \href{\thefield{url}}{\bibtitlehighlight{#1}}%
    }%
  }{%
    \href{http://dx.doi.org/\thefield{doi}}{\bibtitlehighlight{#1}}%
  }%
}
% Use the above macro style for all online references.
\DeclareFieldFormat[article,thesis,report,inproceedings,incollection,book,inbook,techreport,misc,online]{title}{\usebibmacro{string+doiurlisbn}{#1}}
% Disable URLs in online publications.
\DeclareFieldFormat[online]{url}{}
%
% You can add special case handling for certain bibliography items. In the example, we look
% for an entry named 'SomeSpecialReference2008' and add a \linebreak manually.
\renewbibmacro{begentry}{%
  \iffieldequalstr{entrykey}{SomeSpecialReference2008}%
    %{}%
    {\renewcommand{\bibpagespunct}{\addcomma\linebreak\addspace}}%
    {}%
}%

%
% This loads the bibliography from references.bib file, independently whether
% biber or bibtex is used.
\makeatletter
\@ifpackageloaded{biblatex}{\addbibresource{references.bib}}{\bibliography{references}}
\makeatother

%%%%%%%%%%%%%%%%%%%%%%%%%%%%%%%%%%%%%%%%%%%%%%%%%%%%%%%%%%%%%%%%%%%%%%%%%%%%%%
% Nomenclature.
\usepackage[intoc,refpage]{nomencl}
\setlength{\nomitemsep}{-\parsep}
% Use a different title for the nomenclature
\renewcommand{\nomname}{Notation Index}
% Avoids printing the page link when the page does not have an integer number.
% This happens for the nomenclature.tex included before the document begins.
% It will have Letter A or i, because it is placed in the front matter. thefield
% entries in this part are supposed to be general notation which is not
% explicitly defined in the text of the thesis. For example, N, the set of
% natural numbers, which should be clear in any advanced topic.
\def\pagedeclaration#1{\IfInteger{#1}{\hfill\hyperlink{page.#1}{\color{\mainColor}\nobreakspace#1}}{}}
\setlength\nomlabelwidth{2cm}

%%%%%%%%%%%%%%%%%%%%%%%%%%%%%%%%%%%%%%%%%%%%%%%%%%%%%%%%%%%%%%%%%%%%%%%%%%%%%%
% Generally useful packages
%
% Make decisions
\usepackage{ifthen}
%
% Have TODO information added to the thesis (and remove them all at once!)
% When used with TikZ externalize, all todo notes will result in pictures!
% This has to be modified after TikZ has been loaded. See below for the
% redefinition of the \todo command.
\usepackage[disable]{todonotes}



%%%%%%%%%%%%%%%%%%%%%%%%%%%%%%%%%%%%%%%%%%%%%%%%%%%%%%%%%%%%%%%%%%%%%%%%%%%%%%
% Graphics using TikZ
%
\usepackage{tikz}
%
% A list of libraries that are important.
\usetikzlibrary{calc} % coordinate calculations
\usetikzlibrary{decorations}
\usetikzlibrary{intersections}
\usetikzlibrary{matrix}
\usetikzlibrary{external}
\usetikzlibrary{patterns}
\usetikzlibrary{plotmarks}
\usetikzlibrary{fadings}
\usetikzlibrary{decorations.markings}
\usetikzlibrary{arrows}

%
% Plotting
\usepackage{pgfplots}
\pgfplotsset{compat=1.11}
\usepgfplotslibrary{fillbetween}


% Notice: enabling a sub directory here instead of writing to the original
% figures directory, crashes the live preview in Kile. The problem is with
% the -output-directory flag that points to some temp node. A workaround is
% to use the read build (XeLaTex) which produces the .pdf files and then then
% the preview does not have to rebuild the image files.
%\tikzexternalize[figures/compiled/,up to date check=md5]
\tikzexternalize[prefix=figures/compiled/,up to date check=diff]
\tikzset{external/system call={xelatex \tikzexternalcheckshellescape
   -output-driver="xdvipdfmx -V 4" -halt-on-error -interaction=batchmode -jobname "\image" "\texsource"}}
\tikzset{external/only named=true}


% Load todonotes and avoid externalizing the notes by externalize.
\makeatletter
\renewcommand{\todo}[2][]{\tikzexternaldisable\@todo[#1]{#2}\tikzexternalenable}
\makeatother

%%%%%%%%%%%%%%%%%%%%%%%%%%%%%%%%%%%%%%%%%%%%%%%%%%%%%%%%%%%%%%%%%%%%%%%%%%%%%%
% Algorithms
% Install in Ubuntu: algorithm2e sudo apt-get install texlive-science
% Requires packages: etoolbox
%
\usepackage[ruled,algochapter,dotocloa]{algorithm2e}
%
% Define input and output keywords
\makeatletter
\renewcommand{\SetKwInOut}[2]{%
  \sbox\algocf@inoutbox{\KwSty{#2}\algocf@typo:\hspace{1ex}}%
  % define the dimension of the box
  \expandafter\ifx\csname InOutSizeDefined\endcsname\relax% if first time used
    \newcommand\InOutSizeDefined{}\setlength{\inoutsize}{\wd\algocf@inoutbox}%
    \sbox\algocf@inoutbox{\parbox[t]{\inoutsize}{\KwSty{#2}\algocf@typo:\hfill}~}
    \setlength{\inoutindent}{\wd\algocf@inoutbox}%
  \else% else keep the larger dimension
    \ifdim\wd\algocf@inoutbox>\inoutsize%
      \setlength{\inoutsize}{\wd\algocf@inoutbox}%
      \sbox\algocf@inoutbox{\parbox[t]{\inoutsize}{\KwSty{#2}\algocf@typo:\hfill}~}
      \setlength{\inoutindent}{\wd\algocf@inoutbox}%
    \fi%
  \fi%
  % redefine the algorithm command
  \algocf@newcommand{#1}[1]{%
    \ifthenelse{\boolean{algocf@inoutnumbered}}{\relax}{\everypar={\relax}}%
    {
      \vspace{1mm}
      \let\\\algocf@newinout\hangindent=\inoutindent\hangafter=1\parbox[t]{\inoutsize}{\KwSty{#2}\algocf@typo:\hfill}~##1
      \par
      \vspace{1mm}
    }%
    \algocf@linesnumbered% reset line numbers
  }}%
\makeatother
%
% Initialize the new keywords.
\newcommand{\initalgo}{
  \SetKwInOut{Input}{Input}
  \SetKwInOut{Output}{Output}
}
\initalgo
%
% Redefine the algorithm environment to have horizontal lines around the
% headline and at the bottom.
\makeatletter
\renewcommand{\algocf@caption@boxruled}{%
  \hrule
  \hbox to \hsize{%
    \vrule\hskip-0.4pt
    \vbox{
      \vskip\interspacetitleboxruled%
      \unhbox\algocf@capbox\hfill
      \vskip\interspacetitleboxruled
    }%
    \hskip-0.4pt\vrule%
  }\nointerlineskip%
}%
\makeatother

%%%%%%%%%%%%%%%%%%%%%%%%%%%%%%%%%%%%%%%%%%%%%%%%%%%%%%%%%%%%%%%%%%%%%%%%%%%%%%
% Listings.
% Sometimes a real source code listing is required, and not an algorithm.
\usepackage{listings}
\lstset{%
  % Small font
  basicstyle=\small\ttfamily,
  % The default language. Can be modified on each code block
  language=[LaTeX]{TeX}
}

%%%%%%%%%%%%%%%%%%%%%%%%%%%%%%%%%%%%%%%%%%%%%%%%%%%%%%%%%%%%%%%%%%%%%%%%%%%%%%
% General style
%
% allow figures to appear only after the first reference
\usepackage{flafter}


%%%%%%%%%%%%%%%%%%%%%%%%%%%%%%%%%%%%%%%%%%%%%%%%%%%%%%%%%%%%%%%%%%%%%%%%%%%%%%
% Define environments (Theorem, Lemma, etc)
% the following uses one counter only for definitions, theorems and lemmas
% counters are reset by the section
% definition theorem style will use italic font for normal text
\usepackage{amsthm}
\allowdisplaybreaks
\makeatletter
\newtheoremstyle{thesisplain}% name of the style
  {3pt}% space above
  {3pt}% space below
  {\itshape}% body font
  {}% indent
  {\color{\mainColor}\semiboldsans}% header font
  {.}% including dot at end of header
  {.5em}% and some space after header
  {{\thmname{#1}\thmnumber{\@ifnotempty{#1}{ }#2}}%
   \thmnote{ {\sffamily\itshape\the\thm@notefont(#3)}}}%
\newtheoremstyle{thesisdefinition}% <name>
  {3pt}% <Space above>
  {3pt}% <Space below>
  {}% <Body font>
  {}% <Indent amount>
  {\color{\mainColor}\semiboldsans}% <Theorem head font>
  {.}% <Punctuation after theorem head>
  {.5em}% <Space after theorem heading>
  {{\thmname{#1}\thmnumber{\@ifnotempty{#1}{ }#2}}%
   \thmnote{ {\sffamily\itshape\the\thm@notefont(#3)}}}%
\makeatother

\usepackage{thmtools}

\theoremstyle{thesisdefinition}
\declaretheorem[style=thesisdefinition,name=Definition,numberwithin=chapter,qed=$\triangleleft$]{definition}

% Define several numbered environments.
\theoremstyle{thesisplain}
\newtheorem{theorem}[definition]{Theorem}
\newtheorem{corollary}[definition]{Corollary}
\newtheorem{lemma}[definition]{Lemma}
\newtheorem{observation}[definition]{Observation}
\newtheorem{example}[definition]{Example}
