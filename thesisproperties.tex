% PhD thesis template
% © Jan-Philipp Kappmeier, Apache License 2.0
%%%%%%%%%%%%%%%%%%%%%%%%%%%%%%%%%%%%%%%%%%%%%%%%%%%%%%%%%%%%%%%%%%%%%%%%%%%%%
% Mandatory information for any thesis.
%
% The title of the thesis.
\def\thesistitle{The Possibly Long Title of the Thesis}
%
% Used in the covers, if the title is too long for a single line.
\def\thesistitlea{The Possibly Long}
\def\thesistitleb{Title of the Thesis}
%
% An optional sub title.
\def\thesissubtitle{And a Subtitle}
%
% The date of the thesis, e.g. the date of the defense.
\def\thesisdate{14.\,03.\,1526}
%
% The author. You, sir ;-)
\def\thesisauthor{Author Name}
%
% The place.
\def\thesisplace{Troy}
%
% Optional dedication of the thesis.
\def\thesisdedication{for my dog.}
%
% Optional comma separated list of key words. This will be used for the pdf
% metadata entry keywords.
\def\thesiskeywords{topics, and more}
%
% Optional subject of the thesis for the corresponding pdf metadata entry.
\def\thesissubject{}
%
% Type of the thesis
\def\thesistype{Dissertation}



%%%%%%%%%%%%%%%%%%%%%%%%%%%%%%%%%%%%%%%%%%%%%%%%%%%%%%%%%%%%%%%%%%%%%%%%%%%%%
% Mandatory information for the title page in this template, which also
% will be required by most institution's titles.
%
% Institution
\def\institution{Institution}
%
% The chair of the defense committee, not examining but leading the process.
\def\chairman{Gottfried W. Leibniz}
%
% The supervisor of the thesis, typically your professor.
\def\supervisor{Alan M. Turing}
%
% A second examiner of the thesis, typically from an external institution.
\def\examiner{Archimedes von Syrakus}





%%%%%%%%%%%%%%%%%%%%%%%%%%%%%%%%%%%%%%%%%%%%%%%%%%%%%%%%%%%%%%%%%%%%%%%%%%%%%
% What is seen on the cover page? Will be printed on the verso of the 
% bastard title. This place also typically contains a frontispiece.
\def\coverimagedescription{Supporting image to demo thesis template.}





%%%%%%%%%%%%%%%%%%%%%%%%%%%%%%%%%%%%%%%%%%%%%%%%%%%%%%%%%%%%%%%%%%%%%%%%%%%%%
% Information about the published thesis. Most of the properties are optional
% and are only required (and available) after the thesis was graded and is
% ready to be published.
%
% The backtitle contains bibliographic information about the thesis.
% Optional, leave empty for preliminary versions. The information is about
% the thesis, i.e. in which catalogue it is listed. Depending on the cuntry
% and the mode of publication it may be mandadtory to add something. The
% template text is for Germany.
\def\thesisbibliography{\textbf{Bibliographische Informationen der Deutschen Nationalbibliothek}\\
Die Deutsche Nationalbibliothek verzeichnet diese Publikation in der Deutschen Nationalbibliografie; detaillierte bibliografische Daten sind im Internet über \url{http://dnb.d-nb.de} aubrufbar.}
%
% The date of the final revision of the thesis including corrections after
% the defense. Optional.
\def\finalrevision{02.\,07.\,1828}
%
% The publisher which publishes the final thesis.
% Optional, leave empty for preliminary versions.
\def\thesispublisher{epubli GmbH, Berlin, \url{www.epubli.de}}
%
% The isbn number of the final published thesis.
% Optional, leave empty for preliminary versions.
\def\theisbn{726-2-4641-9538-7}
%
% Year of final publication and also copyright.
% Non-optional.
\def\thesisyear{1337}





%%%%%%%%%%%%%%%%%%%%%%%%%%%%%%%%%%%%%%%%%%%%%%%%%%%%%%%%%%%%%%%%%%%%%%%%%%%%%
% Additional information which is good style to include. However, these
% properties are entirely optional. They all belong to the front matter and
% describe technical tools and methods helping the process of writing.
%
% As an alternative to using a traditional colophon at the end of the thesis
% short information is added at the title verso, which I believe is the
% standard nowadays.
%
% Information about the fonts used in the thesis.
\def\colophonfonts{Adobe Minion Pro, Adobe Myriad Pro, typoma Minion Math}
%
% Information about the typesetting used to produce thethesis.
\def\colophontypesetting{XeLaTex, KOMA-Script, TikZ, PGFPlots}
%
% Additional software components, thank you notices or information of the short colophon replacement.
\def\colophonaddendum{This thesis would not have been possible without the many great free and open source projects, especially the work by Donald Knuth and Till Tantau.}
